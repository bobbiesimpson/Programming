\chapter{OpenMP/MPI}

\section{MPI}

\begin{verbatim}
int mpierr, proc, rank;

int MPI_INIT(int *argc, char **argv);	 // initalise the MPI runtime environment - must be called


// do some parallel stuff

int MPI_Finalize(void);	// finish everything
\end{verbatim}

\begin{itemize}
\item We must realise the the processes are totally disconnected - we don't know which order they will come back. 
\end{itemize}

\section{Compiling}

\begin{verbatim}
mpiicc -o helloc hello.c
\end{verbatim}

\section{Running}

\begin{verbatim}
mpirun -np 8 helloc // run with 8 processes
\end{verbatim}

When it returns, execution is completed.

We can then run  

\section{Domain decomposition}

The idea is that we divide the work into smaller chunks which are done on individual processors. We then combine these results to end up with the global result. 

We can use tags to distinguish between messages. There is also NULL type which will not do anything if it received - the message is discarded. 

We can also use a technique called ``probing''. Useful to determine message tag/source.

We should make good use of the MPI_PROC_NULL which basically means that nothing is done when this is received or is sent. 

